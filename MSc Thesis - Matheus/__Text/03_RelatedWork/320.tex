% !TEX root = ./../../_Thesis.tex

% section's Name and Label
\section{Estimating/Correcting Visual Optical Aberrations}
\label{sec:EstimatingCorrectingVisualOpticalAberrations}

\citet{Pamplona2010} presented a practical approach for estimating low-order aberrations without the need of expensive equipments. It uses a pinhole mask attached to a smartphone displaying patterns to the subject. The aberrations are estimated by the subjective alignment of the different patterns.
\citet{Kronbauer2011} developed a psychophysical approach for vision measurement in candelas. It consists in presenting light stimulus in a display in order to discover the absolute threshold for clear and dark conditions. Then, by relating it with an objective vision's assessment (\eg, vision chart acuity and aberrometry data), they have stated a strong correlation between aberrometry data and the absolute threshold.

Many methods have achieved the goal of free the viewer from needing wearable optical correction when looking at displays \cite{Huang2012, Pamplona2012b, Huang2014}, and printings or projections \cite{Montalto2015}. Other works have explored physiologically-based models to provide insights and feedback on how to produce high-fidelity effects and improve visualization experiences \cite{Machado2009, Pamplona2009, Pamplona2011}.  

%\textcolor{red}{In fact, this kind of models has been widely used for different applications, such as 3D eye-model generation \cite{Berard2014}, improvements in individuals' accessibility \cite{Flatla2011}, and biometric authentication based on the iris \cite{Yano2012}. (?)}