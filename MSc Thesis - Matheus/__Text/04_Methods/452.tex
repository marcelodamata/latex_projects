% !TEX root = ./../../_Thesis.tex

% section's Name and Label
\section{Participants}
\label{subsec:Participants}

The inclusion criteria for the study are quite inclusive and only stipulate that all participants must be able to perform subjective and objective refraction test, as well as auto-refraction. 23 individuals meeting the inclusion criteria were asked to to participate. Three had availability restrictions and were not included in the experiment. 

The subjects consisted of 17 males and 3 females, with ages ranging from 23 to 33 years old (mean of 25.3 and standard deviation of 2.51). Out of these, 6 male and the 3 female use corrective eyeglasses, and 1 male uses contact lenses. For these twenty individuals, the minimum, maximum, mean, and standard deviation of their spherical error $S$ in diopters were: $S_{min}$ = -3.5, $S_{max}$ = 1.25, $S_{mean}$ = -0.275, and $S_{std}$ = 1.452, respectively.  Likewise, the minimum, maximum, mean, and standard deviation of their cylindrical error $C$ in diopters were: $C_{min}$ = -2.5, $C_{max}$ = 0.0, $C_{mean}$ = -0.625, and $C_{std}$ = 0.655, respectively. Thus, their spherical equivalent refraction (SER) can be summarized as: $SER_{min}$ = -3.75, $SER_{max}$ = 1.25, $SER_{mean}$ = -0.5877, and $SER_{std}$ = 1.429. The axis of astigmatism of these subjects, expressed in degrees, can be summarized as: $A_{min}$ = 0.0, $A_{max}$ = 179.0, $A_{mean}$ = 57.25, and $A_{std}$ = 69.76.
All these measurements were computed from the results of the autorefrator with the use of cycloplegic eyedrops. 

 