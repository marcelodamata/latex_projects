% !TEX root = ./../../_Thesis.tex

% section's Name and Label
\section{Quasi-Random Algorithm}
\label{subsec:QuasiRandomAlgorithm}

The algorithm developed to determine the absolute threshold is divided into two major phases. In the first one, the goal is to find a tight interval containing the absolute threshold as quick as possible. While this can be efficiently performed using a binary search, we adopted a quasi-random strategy to avoid bias among repetitions of the test.
%limit the intensity range as minimum and fast as possible. 
%The adopted approach guarantees that, in the best case, a single interaction is sufficient to discover the unknown range. 
In the second phase, the participant uses a slider to precisely determine his/her absolute threshold. The second step could be replaced by a binary search.

%\review{verificar para usar pacote algorithm}

\begin{lstlisting}[caption = {Estimating the absolute threshold for vision}]
set minimum intensity to zero
set maximum intensity to one
% Phase 1: determining a tight interval for the absolute threshold 
do {
	Randomly get a new intensity value 
	      inside [minimum intensity, maximum intensity], 
				in multiples of 0.1
	turn on a single pixel in the device screen
	ask if the user see the stimulus
	
	if stimulus is visible
		maximum intensity = new intensity
	else
		minimum intensity = new intensity
	
} while ((maximum intensity - minimum intensity) > 0.1)
% Phase 2: absolute threshold determination
< user interaction based on a slider >
\end{lstlisting}